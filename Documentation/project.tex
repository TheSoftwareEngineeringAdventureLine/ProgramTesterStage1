% !TEX root = SystemTemplate.tex


\chapter{Project Overview}
This section provides some housekeeping type of information with regard to the 
team, project, etc. 



\section{Team Members and Roles}
The team consists of Erik Hattervig, Andrew Koc, and Jonathon Tomes. Erik Hattervig is the Product Owner, Andrew Koc is the Technical Lead, and Jonathon Tomes is the Scrum Master. Erik is responsible for understanding the overall expectations of the product and communication with the customer about specific details regarding the operation and design of the product. Andrew is responsible for designing the technical aspects of the code. Jonathon is responsible for managing meetings and communication between team members and making sure the project is on schedule.


\section{Project  Management Approach}
%This section will provide an explanation of the basic approach to managing the 
%project.  Typically, this would detail how the project will be managed through 
%a given Agile methodology.  The sprint length (i.e. 2 weeks) and product backlog 
%ownership and location (ex. Trello) are examples of what will be discussed.  An 
%overview of the system used to track sprint tasks, bug or trouble tickets, and 
%user stories would be warranted.

	The sprint length for this project was 2 weeks. We began with a meeting to
decide the user needs and split the program accordingly. Each of us would code
different parts of the program and then we would all test and re-code as needed.

	The code was stored, backed up, and shared through git hub. The back log and ownership 
was tracked through Trello. The user stories were condensed and placed on Trello to help 
design break points to split up the program between team members. 

\section{Phase  Overview}
%If the system will be implemented in phases, describe those phases/sub-phases (design, 
%implementation, testing, delivery) and the various milestones in this section. 
 %This section should also contain a correlation between the phases of development 
%and the associated versioning of the system, i.e. major version, minor version, 
%revision. 

The first phase of this Testing program was just to begin working on the program.
The main purpose was to get to receive a root directory, find a .cpp file
in the root. It would then write a log file that starting with a time stamp to
be used later to record the results of tests.
	
	After that it would a crawl through the sub directories recursively
starting at the root, looking for .tst files that would be test cases for 
the program. Along with these would be .ans files that would allow us to
compare the program output and see wich test cases failed. 
	
	It would then out put the results of each test to a log file. With a final
log write that writes the percentage of passed and failed tests.

\section{Terminology and Acronyms}
%Provide a list of terms used in the document that warrant definition.  Consider 
%industry or domain specific terms and acronyms as well as system specific.
none. 
