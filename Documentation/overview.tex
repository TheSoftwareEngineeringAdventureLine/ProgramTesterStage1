% !TEX root = SystemTemplate.tex

\chapter{Overview and concept of operations}

%The overview should take the form of an executive summary.  Give the reader a feel 
%for the purpose of the document, what is contained in the document, and an idea 
%of the purpose for the system or product. 


\section{Scope}
%What scope does this document cover? 
The scope of this document is meant to cover the process, organization,
technologies, and documentation for the creation of the program
Program Tester (stage 1)

\section{Purpose}
%What is the purpose of the system or product? 
The purpose of this program is to run test files on another
 program and record the results in a log file.

It accomplishes this by being given the file path
 to the root directory where a cpp file is present
 along with any test cases in tst files. The tst files
 may be in sub directories starting in the given root
 directory. As tests are run it will write the results
 of the tests to a log file in the root directory.

%\subsection{Major System Component \#1}
%Describe briefly the role this major component plays in this system. 

%\subsection{Major System Component \#2}
%Describe briefly the role this major component plays in this system. 

%\subsection{Major System Component \#3}
%Describe briefly the role this major component plays in this system. 

\section{Systems Goals}
%Briefly describe the overall goals this system plans to achieve.  These goals are 
%typically provided by the stakeholders.  This is not intended to be a detailed 
%requirements listing.  Keep in mind that this section is still part of the Overview. 
1) Locate cpp file and compile it.
 
 2) find test files and run them on the program.
 
 3) record the results.

%\section{System Overview and Diagram}
%Provide a more detailed description of the major system components without getting 
%too detailed.  This section should contain a high-level block and/or flow diagram 
%of the system highlighting the major components.   See Figure~\ref{systemdiagram}.    This is a floating figure environment.  \LaTeX\ will try to put it close %to where it was typeset but will not allow the figure to be split if moving it can not happen.   Figures, tables, algorithms and many other floating %environments are automatically numbered and placed in the appropriate type of table of contents.  You can move these and the numbers will update %correctly.

%\begin{figure}[tbh]
%\begin{center}
%\includegraphics[width=0.75\textwidth]{./diagram}
%\end{center}
%\caption{A sample figure .... System Diagram \label{systemdiagram}}
%\end{figure}

\section{Technologies Overview}
%This section should contain a list of specific technologies used to develop the 
%system.  The list should contain the name of the technology, brief description, 
%link to reference material for further understanding, and briefly how/where/why 
%it was used in the system.    See Table~\ref{somenumbers}.  This is a floating table environment.  \LaTeX\ will try to put it close to where it was typeset but %will not allow the table to be split.   
%\begin{table}[tbh]
%\begin{center}
%\begin{tabular}{|r|l|}
  %\hline
  %7C0 & hexadecimal \\
  %3700 & octal \\ \cline{2-2}
  %11111000000 & binary \\
  %\hline \hline
  %1984 & decimal \\
  %\hline
%\end{tabular}
%\caption{A sample Table ... some numbers. \label{somenumbers}}
%\end{center}
%\end{table}
C++ and g++ were used to create and compile the code in which the program was made.
 To create backups of both the code and documentation Github was used.
 To track user stories and cases as well as divide and keep track of labor Trello was used.
